\section{Zielsetzung}
Im allgeminen soll das Relaxationsverhalten eines RC-Schwingkreises untersucht werden.
Dabei wird die Phasenabhängigkeit des Schwingkreises beobachtet und überprüft ob der RC-Schwingkreis als Integrator fungieren kann.

\section{Theorie}
\label{sec:Theorie}
\subsection{Die allgemine Relaxationsgleichung}
Wenn ein System ausgelenkt wird und nicht oszillatorisch in seinen Anfangszustand zurückkehrt, treten Relaxationserscheinungen auf.
Die Geschwindigkeit der Rückkehr ist dabei proportional zu der Auslenkung:
\begin{equation}
  \frac{dA}{dt}=c[A(t)-A(\infty)]
\end{equation}
Durch Integration von $0$ bis $t$ ergibt sich:
\begin{equation}
  \ln{\frac{A(t)-A(\infty)}{A(0)-A(\infty)}}=ct
\end{equation}
Wird die e-Funktion auf die Gleichung anngewendet ergibt sich:
\begin{equation}
  \label{eq:gl2}
  A(t)=A(\infty)+[A(0)-A(\infty)]\exp(ct)
\end{equation}
Wobei $c<0$ sein muss, damit $A$ beschränkt ist.
\subsection{Anwendung auf den Auf- und Entladevorgag des RC-Schwingkreises}
Der in Abbildung befindliche Kondensator soll aufgeladen sein, dann liegt zwischen den Platten eine Spanung
\begin{equation}
  U_C=\frac{Q}{C}
\end{equation}
an.
Nach dem ohmschen Gesetz lässt sich der Strom durch
\begin{equation}
  I = \frac{U_C}{R}
\end{equation}
ausdrücken.
Damit findet sich für den zeitlichen Verlauf der Ladung folgende Dgl.:
\begin{equation}
  \frac{dQ}{dt}=\frac{1}{RC}Q(t)
\end{equation}
Mit $Q(\infty)=0$ ergibt sich analog zu der Gleichung\eqref{eq:gl2}:
\begin{equation}
  Q(t)=Q(0)\exp(\frac{-t}{RC})
\end{equation}
Für den Aufladevorgang gelten die Randbedingungen
\begin{equation}
  Q(0)=0
\end{equation}
und
\begin{equation}
  Q(\infty)=CU_0 .
\end{equation}
Damit folgt für den Zeitlichhen Verlauf der Ladung:
\begin{equation}
  Q(t) = CU_0 \exp(\frac{-t}{RC})
\end{equation}
Die Zeitkonstante ist ein Maß für die Geschwindigkeit der Relaxation des Systems. Hier ist diese $\frac{1}{RC}$.

\subsection{Auf- und Entladevorgag mit periodischer Anregung}
Liegt eine Wechselspannung
\begin{equation}
  U(t)=U_0 cos(\omega t)
\end{equation}
an, so lässt sich mit folgendem Ansatz eine Lösung für das Problem finden:
\begin{equation}
  U_c(t)= A(\omega) \cos{(\omega t + \phi(\omega))}
\end{equation}
Damit gilt für den Stromkreis, unter einbezug des zweiten Kirchhoffschen Gesetzes:
\begin{equation}
  \label{eq:gl3}
  U_0 \cos{(\omega t)} = A \omega R C \sin{(\omega t + \phi)} + A(\omega)\cos{(\omega t + \phi)}
\end{equation}
Gleichung \eqref{eq:gl3} muss für alle $t$ gelten. Mit $\omega t= \frac{\pi}{2}$ ergibt sich dann:
\begin{equation}
  0 = - \omega R C \sin{(\frac{\pi}{2}+\phi)} + \cos{(\frac{\pi}{2}+\phi)}
\end{equation}
Durch umformung ergibt sich dann folgende Beziehung für die Phasenverschiebung:
\begin{equation}
  \phi(\omega)=\arctan{(- \omega R C)}
\end{equation}
Mit $\omega +\phi =\frac{\pi}{2}$ ergibt sich für die Generatorspannung:
\begin{equation}
  \label{eq:gl4}
  A(\omega)= \frac{U_0}{\sqrt{1+\omega^2 R^2 C^2}}
\end{equation}
Es ist durch Gleichung\eqref{eq:gl4} erkennbar, dass das RC-Glied ein Tiefpass ist.

\subsection{Der RC-Kreis als Integrator}
Es gilt:
\begin{equation}
  U(t)= RC \frac{dU_c}{dt} + U_c(t)
\end{equation}
Unter der Voraussetzung $\omega \gg\frac{1}{RC}$ ist $|U_C| \ll|U|$.
Somit lässt sich näherungsweise
\begin{equation}
  U(t)=RC \frac{dU_C}{dt}
\end{equation}
schreiben.
Anders lässt sich dies als
\begin{equation}
  U_C(t)=\int_0^t U(t') dt
\end{equation}
schreiben.
Die am Kondensator anliegende Spannung ist also proportional zu dem Integral der Generatorspannung.
